\chapter{Architettura di un sistema operativo}
Un principio importante \`e la separazione tra meccanismi e criteri o policy: i primi determinano come eseguire qualcosa, mentre i secondi cosa si deve fare. Questa distinzione \`e importante ai fini della
flessibilit\`a in quanto i criteri sono soggetti a cambiamenti di luogo e tempo.Principi importanti da tenere a mente durante lo sviluppo di sistemi operativi \`e il KISS (keep it small
and simple), semplice dal punto di vista del codice, per mantenere affidabilit\`a e mantenibilit\`a e il POLA (principle of least privilege): un programm deve poter accedere unicamente ai
dati strettamente necessari, fondamentale per il mantenimento di sicurezza e affidabilit\`a.Questi cambiamenti devono richiedere il cambio di meccanismi solo nel caso pessimo. Le 
principali tipologie di architettura di sistemi operativi sono:
\begin{itemize}
	\item Sistemi monolitici: sistemi senza gerarchia e con un unico strato software tra utente e hardware, tutti i componenti sono sullo stesso 
		livello e possono chiamarsi vicendevolmente. In questa tipologia il codice dipende direttamente dall'architettura hardware e rende test
		e debugging complesso.
	\item Sistemi a struttura semplice: si ha un minimo di gerarchia e di struttura, non esiste  ancora la suddivisione modalità utente e modalità kernel. Questa struttura \`e mirata a ridurre i costi di sviluppo e manutenzione.
	\item Sistema operativo a livelli. I servizi sono organizzati su livello gerarchici con al livello pi\`u alto l'interfaccia utente e al pi\`u 
		basso l'hardware. Ogni livello pu\`o utilizzare funzioni di livelli inferiori. La modularit\`a rende pi\`u semplice la manutenzione, ma
		diminuisce l'efficienza e richiede un'attenta definizione dei livelli. 
	\item Sistemi basati su kernel: vengono utilizzati due livelli, i cui servizi sono distinti tra kernel e non-kernel. Presenta i vantaggi del sistema a livelli come modularit\`a
		ma senza avere troppi livelli. Tra i servizi al di fuori del kernel non si trova nessuna organizzazione e si tende a pensare al kernel come a una struttura monolitica.
	\item Sistemi a micro-kernel: i micro-kernel sono un insieme di piccoli kernel che svolgono poche funzioni fondamentali. Occupano meno memoria e sono pi\`u affidabili e 
		mantenibili, ma presentano scarse prestazioni: ogni volta che si deve accedere ad un programma applicativo si deve fare un cambio tra modalit\`a kernel a modalit\`a
		utente e viceversa una volta terminato il processo. Vengono utilizzati da quando le prestazioni di CPU e memoria sono sufficienti a non far percepire all'utente il cambio
		di modalit\`a. La modularit\`a offre inoltre maggiore sicurezza e portabilit\`a.
\end{itemize}
\section{macchine virtuali}
Le macchine virtuali sono introdotte nel $1972$ da IBM come estremizzazione dell'approccio a livelli persata per offrire un sistema di timesharing multiplo ovvero che permette la
multiprogrammazione e una macchina estesa che abbia un interfaccia pi\`u semplice del solo hardware. La base della macchina virtuale \`e la separazione di questi due aspetti. La sua parte
centrale era il virtual machine monitor che permette la multiprogrammazione offrendo diverse macchine virtuali al livello superiore. Un tipo di macchina virtuale utilizza un type 1 
hypervisor, usato comuniemente che si trova sull'hardware e permette di eseguire diversi sistemi operativi sulla stessa macchina. Il type 2 hypervisor viene utilizzato su un sistema 
operativo host nel quale l'hypervisor installa il sistema operativo guest in un disco virtuale che il sistema host vede come un file di grandi dimensioni. La differenza tra i due tipi
di hypervisor sta nel fatto che quello di tipo 1 si trova direttamente sull'hardware, mentre il tipo 2 viene creato in un sistema operativo host.
\section{Processi e thread}
Attributi (Process Control Block): contiene un puntatore alla cella di memoria che contiene l’immagine, contiene lo stato del processo in un determinato momento, contiene i registri, le informazioni relative allo stato dell’I/O. 
Un processo può essere in diversi stati. 
All’inizio il processo viene creato, poi può essere in esecuzione se gli viene assegnata la CPU o non in esecuzione se non gli viene assegnata la CPU. Il Dispatcher assegna la CPU ai processi che sono pronti, ma non in esecuzione (posso 
avere diversi processi in memoria, ma solo uno per volta usa la CPU e quindi è effettivamente in esecuzione, a meno di CPU multicore). Quando un processo è pronto e viene creato viene messo nella ReadyQueue (oppure nella coda di un 
dispositivo cioè la coda in cui viene messo un processo che sta aspettando di accedere a un determinato dispositivo). In realtà esistono diverse code in cui può essere messo un processo. Dispatch e Scheduler sono due componenti diversi, 
lo scheduler sceglie mentre il dispatcher implementa questa cosa. Context-Switch: salvo tutto quello che c’era nel PCB, tutto quello che stavo utilizzando al momento dell’esecuzione. Due operazioni fondamentali che fa un sistema operativo
è la creazione e la terminazione del processo. Ogni processo può creare altri processi e questi prendono il nome di processi figli. Il figlio può essere creato in modalità sincrona, cioè finché il figlio è in vita io non faccio niente, 
oppure in modalità asincrona cioè io creo il figlio e continuo la mia esecuzione. Con la fork il figlio lo creo esattamente uguale al padre, con la exec posso caricare sul figlio un programma diverso rispetto al padre. Con la wait creo 
un’esecuzione sincrona tra padre e figlio. Una fork può fallire perché o il padre non ha abbastanza memoria o perché non ha i privilegi per creare un figlio. L’exec cambia l’immagine in memoria del figlio. 
