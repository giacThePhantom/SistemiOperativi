\chapter{Il sistema di I/O}
Il sottosistema di I/O \`e l'insieme di metodi per controllare i dispositivi di I/O con l'obiettivo di fornire ai processi utente un'interfaccia efficiente e indipendente dai 
dispositivi. Consiste dell'interazione tra l'hardware di I/O e il software determinato dal sistema operativo. 
\section{Hardware di I/O}
Esistono numerosi ed eterogenei dispositivi di I/O per memorizzare (dischi, nastri), trasmissione (modem, schede di rete) e interazione uomo-macchina (tastiera, monitor). Si deve fare
una distinzione tra dispositivi (non elettronici) e controllo dei dispositivi (elettronici). I concetti comuni per tutti i dispositivi sono:
\begin{itemize}
	\item Porta: punto di connessione.
	\item Bus a daisy chain o condiviso, insieme di fili e relativo protocollo.
	\item Controllore che agisce su porta, bus o dispositivi. 
\end{itemize}
\subsection{Controllore dei dispositivi}
Il controllore \`e la parte elettronica di un dispositivo, detto anche device controller. Viene connesso tramite bus al resto del sistema. \`E associato ad un indirizzo e contiene 
registri per comandare il dispositivo:
\begin{itemize}
	\item Registro di stato: per capire se il comando \`e stato eseguito, se c'\`e stato un errore o se i dati sono pronti per essere letti.
	\item Registro di controllo: per inviare comandi al dispositivo.
	\item Un numero arbitrario di buffer per la conversione dei dati. 
\end{itemize}
L'accesso ai registri pu\`o avvenire in vari modi:
\begin{itemize}
	\item Dispositivi mappati in memoria (memory-mapped): i registri sono visti nello spazio di indirizzamento della memoria e il loro accesso avviene tramite istruzioni di accesso
		alla memoria (necessario disabilitare la cache). Permettono di scrivere driver in linguaggio di alto livello e non esiste nessun speciale accorgimento per la protezione
		in quanto \`e sufficiente allocare lo spazio di indirizzamento fuori dallo spazio utente.
	\item Dispositivi mappati su I/O (I/O-mapped): l'accesso ai registri avviene tramite istruzioni specifiche e non esiste nessun problema di gestione della cache. 
	\item Soluzioni ibride come il Pentium in cui i registri sono I/O mapped e i buffer memory mapped. 
\end{itemize}
Per esempio per il display l'hardware mappa i registri e la memoria del display nello spazio di indirizzi fisico, si scrive nella mem (frame buffer) i cambi che avvengono nello schermo e
le descrizioni grafiche nell'area della coda di comandi. Scrivere nel registro comando potrebbe causare all'hardware grafico del display di fare qualcosa. 
\subsection{Accesso ai dispositivi di I/O}
\subsubsection{Polling}
Determina lo stato del dispositivo mediante lettura ripetuta del busy-bit del registro di status: quando \`e $0$ il comando viene scritto nel registro di controllo e il command-ready-bit
posto a $1$ e l'operazione di I/O viene eseguita. \`E un ciclo di attesa attiva e pertanto c'\`e spreco di CPU. 
\subsubsection{Interrupt}
Il dispositivo di I/O avverte la CPU tramite un segnale su una connessione fisica. Le problematiche sono gestite da controllore di interrupt che possono essere mascherabili (ignorabile
durante l'esecuzione di istruzioni critiche), sono numerati in modo che il valore sia l'indice nel vettore di interrupt che consegna l'interruzione al gestore corrispondente e basato su
priorit\`a. Gli interrupt multipli sono ordinati e gli interrupt con priorit\`a pi\`u alta prelazionano quelli con priorit\`a pi\`u bassa. 
\subsubsection{DMA - direct memory access}
Questo metodo nasce per evitare I/O programmato per grandi spostamenti di dati: usare la CPU per controllare il registro di stato del controllore e trasferire pochi byte alla volta \`e 
uno spreco. Richiede hardware esplicito (DMA controller). \`E basato sull'idea di bypassare la CPU per trasferire dati direttamente tra I/O e memoria. Per effettuare un trasferimento
la CPU invia al DMAC l'indirizzo di partenza dei blocchi/memoria da trasferire, l'indirizzo di destinazione, il numero di byte da traferire e la direzione del trasferimento. Il DMA
controller gestisce il trasferimento comunicando con il controllore del dispositivo mentre la CPU effettua altre operazioni e interrompe la CPU al termine del trasferimento. 
\section{interfaccia di I/O}
I dispositivi di I/O si differenziano per molti aspetti e pertanto si rende necessaria dell'astrazione, incapsulamento e software layering per nascondere le differenze al kernel del 
sistema operativo costruendo un'interfaccia comune che d\`a un insieme di funzioni standard. Le differenze sono incapsulate nei device driver. Le interfacce semplificano il lavoro del
sistema operativo aggiungendo nuovo hardware senza modificare il sistema operativo. 
\subsection{Dispositivi a blocco e a carattere}
Nell'interfaccia dei dispositivi a blocchi i device memorizzano e trasferiscono dati in blocchi, la lettura e scrittura di un blocco \`e indipendente dagli altri, i comandi sono di 
read, write e seek, l'interfaccia \`e tipica nei dischi e il memory-mapped I/O sfrutta il block-device driver. Nell'interfaccia dei dispositivi a carattere i dispositivi memorizzano e trasferiscono le stringhe di
caratteri, non esiste indirizzamento (seek) e i comandi sono del tipo di get e put. I dispositivi tipici sono terminali, mouse e porte seriali. 
\section{Software di I/O}
Gli obiettivi del software di I/O sono l'indipendenza dal dispositivo, la notazione uniforme, gestione degli errori, la gestione di varie opzioni di trasferimento e le prestazione. \`E
organizzato per livelli di astrazione: 
\begin{enumerate}
	\item Gestori degli interrupt: sono astratti il pi\`u possibile dal resto del sistema operativo e bloccano e sbloccano i processi attraverso semafori o segnali.
	\item Device driver: devono tradurre le richieste astratte del livello superiore in richieste device-dependent, contengono tutto il codice device-dependent, condivisi per classi
		di dispositivi, interagiscono con i controllori dei dispositivi e sono tipicamente scritti in linguaggio macchina. 
	\item Software del sistema operativo: indipendente dal dispositivo. Ha funzioni speciali come la definizione di interfacce comuni, naming (individuare un dispositivo e gestire i 
		nomi), protezione (tutte le primitive di I/O sono privilegiate), buffering, ovvero la memorizzazione dei dati durante un trasferimento per gestire diverse velocit\`a 
		e diverse dimensioni. Deve definire la dimensione del blocco, allocare e rilasciare i dispositivi e gestire gli errori tipicamente device-dependent. 
	\item Programmi utente: sono system call per l'accesso ai dispositivi.
	\item Spooling: la gestione di I/O dedicato, non condivisibile. Ad esempio pi\`u processi possono voler scrivere su una stampante contemporaneamente, ma le stampe non devono essere interfogliate, 
pertanto i dati da processare vanno accodati nella spooling directory e lo spooler, un processo di sistema \`e l'unico autorizzato ad accedere alla stampante. Periodicamente stampa i dati nella spooling directory.
\end{enumerate}
