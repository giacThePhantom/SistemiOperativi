\chapter{Protezione e sicurezza}
\subsection{Propriet\`a di sicurezza}
\begin{itemize}
	\item Integrit\`a: garantire che il file non sia stato modificato in modo non autorizzato.
	\item Segretezza e confidenzialit\`a: garantire che il dato possa essere letto solo da chi \`e autorizzato.
	\item Autenticit\`a: che l'utente sia effettivamente chi dice di essere.
\end{itemize}
\section{Primitive di crittografia}
\subsection{Hash crittografiche}
Un'hash crittografica \`e una funzione con un output di lunghezza fissa che mappa stringhe di qualsiasi dimensione a stringhe di lunghezza fissa. Le funzioni di hash crittografiche 
devono essere di facile computazione, deve essere impossibile determinare la preimmagine. In caso di collisione deve essere possibile determinare solo il messaggio reale. Esempi di 
hash sono SHA-512 e Keccak e vengono utilizzate nella distribuzione di software online per controllarne l'integrit\`a. 
\subsection{Crittografia a chiave simmetrica}
Sia \emph{E()} la funzione di crittazione e \emph{D()} la sua inversa funzione di decrittazione. Entrambe oltre al messaggio richiedono la stessa chiave segreta per cifrare e decifrare
il messaggio. Esistono due tipi di cifrari: a flusso (stream) in cui si usa una chiave molto lunga univoca per il messaggio e \emph{E()} \`e una \emph{XOR} e a blocco, in cui \emph{E()}
prende in input una chiave di dimensione fissa di $56$, $128$ o $256$ bit. RC4 viene usato in SSL/TLS e nei browser, A5/3 nei GSM e AES \`e il cifrario a blocco standard. Viene 
utilizzata per cifrare i file systems e il traffico di rete (SSL, IPSEC). 
\subsection{Crittografia a chiave pubblica (asimmetrica)}
Viene inventata da Diffie e Hellman nel $1976$. Ogni utente ha due chiavi diverse: una chiave pubblica per la cifratura e una privata (segreta) per la decrittazione. La chiave pubblica
\`e costruita in modo per cui non si pu\`o derivare da essa la privata ed esiste una corrispondenza unica tra le due. Ogni utente ha una coppia di chiavi: privata e segreta, ogni utente
conosce tutte le chiavi pubbliche e solo il proprietario conosce la sua chiave segreta. Viene utilizzato oltre al file cifrato una firma del file che determina il proprietario del file. 
Esempi di algoritmi a chiave pubblica sono RSA e DSA (basata sulle curve ellittiche per smartcard e ambienti embedded). 
\section{Trusted boot}
Si rende necessario essere sicuri dell'autenticit\`a del kernel caricato in memoria al momento di avvio per garantire integrit\`a ed autenticit\`a. 
\subsection{BIOS}
Il BIOS \`e il Basic Input-Output System. \`E firmware in ROM che fornisce un gruppo di interfacce indipendenti dal sistema operativo per l'hardware:
\begin{itemize}
	\item Int10 per servizi video.
	\item Int13 per servizi di disco.
	\item Int16 per servizi di tastiera.
	\item Int18 il caricatore della ROM BIOS.
	\item Int19 per il caricatore del bootstrap. 
\end{itemize}
L'hardware controlla prima di caricare il programma che carica il sistema operativo. L'interfaccia tra il firmware e il sistema operativo viene fornita da UEFI, unified extensible 
firmware interface. 
\subsection{Trusted platform module (TPM)}
Il TPM \`e un chip sicuro integrato sulla scheda madre che fornisce la generazione e memorizzazione sicura delle chiavi crittografiche implementando le primitive. \`E un coprocessore
trusted separato. 
\section{Cifratura disco}
La partizione di sistema operativo cifrata contiene il sistema operativo, lo swap, i file temporanei, il file system e i file di ibernazione. La chiave di cifratura SRK (storage root
key) \`e memorizzata nella TPM e viene usata per cifrare la chiave di cifratura dell'intero volume (FVEK: full volume encryption key) protetta da un PIN che viene memorizzata cifrata
sul disco nel volume del sistema operativo. 
\section{Login - autenticazione degli utenti}
L'autenticazione si pu\`o basare su quello che l'utente conosce come PIN o password, su quello che ha come USB o smartcard o su quello che l'utente \`e o fa (biometrie fisiologiche o 
comportamentali). L'autenticazione si svolge in due operazioni: l'identificazione in cui l'utente annuncia chi \`e e l'autenticazione vera e propria dove si verifica che \`e chi dice di
essere.
\section{Malware}
Si dice malware un software malevolo che intenzionalmente ha l'obiettivo di violare una o pi\`u propriet\`a di sicurezza.
\subsection{Trojan horse}
Sono programmi che replicano le funzionalit\`a di programmi di uso comune o programmi dall'apparenza innocua ma che contengono codice malevolo. Tipicamente catturano informazioni e le 
inviano al creatore del programma come informazioni critiche per la sicurezza del sistema, private dell'utente e compromettono o distruggono informazioni importanti per il funzionamento
del sistema. 
\subsection{Keylogger}
Sono programmi che registrano ci\`o che l'utente scrive su tastiera e lo inviano all'utente.
\subsection{Ransomware}
Sono programmi che bloccano l'accesso al computer o ai dati e l'utente deve pagare per il loro ripristino. 
\subsection{Rootkits}
Sono programmi installati con privilegi di amministratore e inseriscono logica malevola modificando il sistema operativo host per non essere identificabili. Nascondono la loro esistenza
modificando i normali meccanismi del sistema operativo come le tabelle dei processo, files, registry entries. Possono essere persistenti o caricati ogni volta in memoria come prima 
operazione al boot e girare in user o kernel mode. Sono installati tipicamente via Trojan o da intrusione fisica. Occorrono diverse contromisure per renderli innocui. 
\subsection{Virus}
Si dice virus un software che infetta il programma modificandoli per includere una copia del virus in modo che venga eseguito segretamente quando il programma host \`e eseguito. \`E
specifico al sistema operativo in quanto sfrutta suoi dettagli e debolezze. Un virus tipico si trova nelle fasi dormiente, in propagazione, causazione e esecuzione. Il meccanismo di 
infezione permette la replicazione, il trigger \`e l'evento che rende il payload attivo e il payload \`e ci\`o che il virus fa. Quando il programma infetto viene invocato il codice del 
virus viene invocato e eseguito prima del codice originale. Si pu\`o bloccare l'infezione iniziale o la propagazione. 
\subsection{Worms}
Sono programmi replicanti che si propagano lungo la rete usando email, esecuzioni e login remoti, tipicamente attraverso l'esecuzione arbitraria del codice causata da buffer overflow. 
Ha le stesse fasi di un virus e in quella di propagazione cerca per altri sistemi, si connette ad essi, si copia su di loro, esegue e ripete. Pu\`o nascondersi come sistema di processi.
\section{Vulnerabilit\`a}
Il buffer \`e un blocco di memoria che contiene una o pi\`u istanze di qualche dato, tipicamente associato ad un array. Tipicamente hanno dimensioni predefinite. Il buffer overflow 
accade quando il dato di input ha dimensioni maggiori rispetto alla dimensione del buffer. La vulnerabilit\`a sfrutta il layout in memoria del programma e i registri della CPU. 
La memoria viene divisa in dati e testo: la parte di dati referenzia informazioni su variabili definite a tempo di compilazione e il testo \`e il codice eseguibile del programma. 
Lo stack salva informazioni temporanee, \`e LIFO e cresce verso indirizzi pi\`u bassi e salva l'indirizzo di ritorno dove il programma deve andare finita la subroutine. L'heap contiene
dati allocati a run-time e cresce verso indirizzi di memoria pi\`u alti. Quando chiamate la funzioni sono appese allo stack di memoria creando un nuovo stack frame e i buffer sono 
aree di memoria che sono allocate per conservare dati di input. Un utente malevolo pu\`o scrivere come input una stringa che sovrascriva l'indirizzo di ritorno in modo che esegua codice
che salti al proprio codice (shellcode)
