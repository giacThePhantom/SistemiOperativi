\chapter{Processi e threads}
Il concetto centrale per ogni sistema operativo \'e quello di processo, l'astrazione di un programma che sta venendo eseguito. Tali astrazioni permettono 
di avere operazioni pseudo-concorrenti anche quando esiste una sola CPU, trasformandola in diverse CPU virtuali. 
\section{Processi}
Tutti i computer moderni svolgono diverse funzioni allo stesso tempo. In un sistema multiprogramma la CPU cambia da processo a processo rapidamente, 
eseguendo ognuno per decine o centinaia di millisecondi. Si parla di pseudoparallelismo in contrasto con il parallelismo dei sistemi multiprocessore. .
\subsection{Il modello dei processi}
In questo modello tutti il software eseguibile sul computer \`e organizzato in un numero di processi sequenziali, istanze di un programma che sta venendo eseguito con i valori per 
il contatore, i registri e le variabili. Ogni processo possiede la propria CPU virtuale, anche se \`e la CPU che cambia tra i processi, chiamato multiprogramming. Ogni programma in
questo caso viene eseguito in maniera indipendente. Essendo che esiste un unico contatore di programma fisico, quando un processo viene eseguito il suo contatore logico \`e inserito 
in quello reale. Quando si passa ad un altro processo il contatore fisico \`e salvato nel contatore logico del processo. Si assuma che ci sia una sola CPU. Con essa che cambia tra i 
processi, il tasso di computazione di un processo non sar\`a uniforme n\`e riproducibile, pertanto i processi non possono essere programmati con assunzioni riguardo alla temporizzaizone.
Quando un processo richiede dei criteri real-time eventi si devono prendere delle misure speciali. Il processo pu\`o essere considerato come un'istanza del programma, con un input, un
output e uno stato. Si noti come se un programma viene eseguito due volte conta come due proessi. 
\subsection{Creazione dei processi}
I sistemi operativi necessitano di un modo per creare processi. In sistemi progettati per eseguire una singola applicazione potrebbe essere possibile avere tutti i processi che saranno
necessari presenti allo startup. In sistemi general-purpose \`e neccessario avere qualche modo per creare e terminare processi quando sono necessari durante l'operazione. Ci sono 
quattro princupali cause per la creazione di un processo:
\begin{itemize}
	\item Inizializzazione del sistema. Durante la fase di boot sono creati numerosi processi, alcuni che interagiscono con l'utente e altri di background che non sono associati con
		utenti particolari ma hanno funzioni specifiche. I processi che stanno nel background per gestire delle attivit\`a sono detti daemons e sistemi grossi ne possiedono a 
		dozzine. 
	\item Esecuzione di una system call da un processo che sta venendo eseguito che crea un processo attraverso una system call. Creare un nuovo processo \`e utile quando il lavoro
		che deve essere eseguito pu\`o essere formulato come un insieme di processi che interagiscono ma sono indipendenti. 
	\item L'utente richiede di creare il processo in sistemi interattivi attraverso un comando o cliccando su un'icona. In sistemi basati su UNIX il nuovo processo rileva la finestra
		da dove \`e eseguito. In windows il processo pu\`o creare una o pi\`u finestre. In entrambi i casi l'utente pu\`o avere pi\`u finestre aperte contempoaneamente.
	\item Inizializzazione di un batch job su sistemi batch che si trovano su grandi mainframes. Gli utenti possono inviare batch jobs al sistema e quando il sistema decide che ha
		le risorse necessarie per eseguirne un altro crea un nuovo processo e svolge il job successivo nella coda.
\end{itemize}
Tecnicamente in tutti questi casi un nuovo processo \`e creato avendo un processo esistente che esegue una system call che dice al sistema operativo di creare un nuovo processo e indica
direttamente o indirettamente quale programma eseguire in esso. In UNIX esiste un'unica system call per creare un nuovo processo: \emph{fork} che crea un clone esatto del processo
chiamante. Dopo la \emph{fork} i due processi hanno la stessa immagine di memoria, le stesse stringhe ambientali e gli stessi file aperti. Tipicamente il processo figlio esegue
\emph{execve} o una system call simile per cambiare l'immagine di memoria e eveguire un nuovo programma. Quando l'utente inserisce un comando la shell forka il processo figlio che poi
esegue il comando. QUesti due passaggi permettono al figlio di manipolare i propri file descriptors dopo la \emph{fork} ma prima dell'\emph{execve}. In Windows una singola chiamata alla
funzione \emph{CreateProcess} gestisce entrambi i passaggi con $10$ parametri che includono la creazione e il caricamento del programma corretto da eseguire, i parametri di liena di 
comando, attributi di sicurezza, bit di controllo e un puntatore alla struttura in cui le informazioni sul processo sono ritornate al chiamante. Dopo che un processo \`e creato 
il genitore e il figlio possiedono i propri spazi di indirizzamento distinti. In UNIX quello del figlio \`e inizialmente una copia del genitore ma non \`e condivisa memoria scrivibile.
\subsection{Terminazione di un processo}
Dopo che un processo \`e stato creato e ha svolto il suo lavoro termina in:
\begin{itemize}
	\item Normal exit (volontaria), la maggior parte dei processi termina in questo modo eseguendo una system call in modo da dire al sistema operativo che ha finito. 
		Questa call \`e \emph{exit} in UNIX e \emph{ExitProcess} in Windoes. Programmi screen-oriented supportano la terminazione volontaria. 
	\item Error exit (volontaria), avviene quando il processo scopre un errore, lo annuncia ed esce, le applicazioni screen oriented tipicamente creano una dialog box.
	\item Fatal error (involontaria) \`e causata da un bug nel programma, come l'esecuzione di un'istruzione illegale. 
	\item Ucciso da un altro processo (involontario) attraverso una system call che dice al sistema operativo di terminarlo in UNIX \`e \emph{kill}, in Windows \`e 
		\emph{TerminateProcess}. 
\end{itemize}
In alcuni sistemi quando un processo termina sono uccisi anche tutti i processi che ha creato. 
\subsection{Gerarchie di processi}
In alcuni sistemi quando un processo ne crea un altro rimangono associati in certi modi. in UNIX un progesso e tutti i suoi discendenti formano un gruppo di processi. Quando un segnale
viene inviato dalla tastiera viene ricevuto da tutti i membri del gruppo di processi associati alla tastiera. Individualmente ogni processo pu\`o catturare il segnale, ignorarlo o 
svolgere un'azione default, che \`e di essere ucciso dal sengale. Un altro esempio di gerarchia \`e dato dal inizializzazione di UNIX successivamente la fase di boot. Un processo 
speciale detto \emph{init} \`e presente nella immagine di boot e quando viene eseguito legge un file che dice quanti terminali ci sono. Successivamente forka ad un nuovo processo per 
terminale. Questi processi aspettano per un login che se hanno successo esegue una shell per eseguire comandi che potrebbero iniziarne altri e cos\`i via. Pertanto tutti i processi
del sistema si basano su un singolo albero con \emph{init} alla radice. Windows non possiede un concetto di gerarchia dei processi: sono tutti uguali, ma quando un processo \`e generato
il padre possiede un token detto handle che gli permette di controllare il figlio. Questo tolen pu\`o essere passato ad altri process, annullando la gerarchia. 
\subsection{Stati dei processi}
Nonostante ogni processo sia un'entit\`a indipendente, con il proprio program counter e stato interno, deve spesso interagire con altri processi, ad esempio accettando come input 
l'output di altri. Quando un processo si blocca lo sa in quanto non pu\`o continuare logicamente, tipicamente quando sta aspettando per un input che non \`e disponibile. \`E inoltre 
possibile che sia bloccato in quanto il sistema operativo ha deciso di allocare la CPU per un altro processo. Queste due condizioni sono diverse in quanto la prima \`e inerente il 
problema, mentre nel secondo caso \`e una tecnicalit\`a del sistema. Ci sono pertanto tre stati che il processo pu\`o avere:
\begin{itemize}
	\item Running: il processo sta utilizzando la CPU.
	\item Ready: eseguibile, stoppato per far eseguire un altro processo. Stato logicamente simile al primo in quanto il processo in entrambi i casi \`e capace di essere eseguito.
	\item Blocked: incapace di essere eseguito fino a che un evento esterno accade. Questo stato \`e differente in quanto il processo non pu\`o essere eseguito anche con la CPU
		libera.
\end{itemize}
L'unica transizione non possibile \`e duella da ready a blocked. La transizione da running a blocked avviene quando il sistema scopre che un processo non pu\`o contineare al momento. 
In alcuni sistemi si pu\`o eseguire una system call come \emph{pause} per entrare in uno stato bloccato, in altri sistemi come UNIX, quando un processo legge da una pipe o da un file
speciale e non si trova input disponibile il processo \`e automaticamente bloccato. La transizione da running a ready e viceversa sono causate dallo scheduler dei processi in modo da
permettere l'eventuale esecuzione di tutti i processi in stato ready. La transizione da blocked a running avviene quando un evento esterno elimina il blocco logico del processo. 
Il modello dei processi permette di semplificare gli eventi interni al sistema: alcuni processi eseguono comandi dell'utente, altri sono parte del sistema e gestiscono richieste di 
esso. Quando avviene un interrupt il sistema ferma il processo corrente ed esegue l'interrupt che si bloccano quando aspettano che accada qualcos altro. Il livello pi\`u basso del
sistema operativo \`e lo scheduler, con un insieme di programmi al di sopra di esso. Tutti i dettagli della gestione dell'interrupt e di blocco e inizio dei processi sono nascosti e il
resto del sistema operativo \`e stutturato in forma di processo. 
\subsection{Implementazione dei processi}
Per implementare il modello dei processi il sistema operativo mantiene una tabella (array di strutture) chiamata la tabella dei processi con un entry per processo contenente 
informazioni riguardo lo stato del processo, come il program counter, lo stack pointer, la memoria allocata, lo stato dei file aperti e le informazioni di accounting e scheduling e
ogni altra informazione che deve essere salvata quando il processo viene passato da running a ready o blocked in modo che possa essere fatto ripartire successivamente. In un tipico 
sistema la tabella presenta tre colonne con la prima dedicata alla gestione del processo, la seconda alla gestione della memoria e la terza alla gestione dei file. Associato con ogni
classe di I/O si trova una locazione (tipicamente fissa al fondo della memoria) detta interrupt vector che contiene l'indirizzo dell procedura del servizio di interrupt. Quando accade
un interrupt tutti i processi che stanno venendo eseguiti salvano il program coutner, lo stato e dei registri nello stack grazie all'hardware di interrupt e il computer salta 
all'indirizzo specificato nell'interrupt vector che fa partire la procedura. Tutti gli interrupt cominciano con il salvataggio dei registri nell'entri del processo corrente, dopo 
l'informazione \`e pushata sullo stack dall'interrupt \`e rimossa e il puntatore allo stack \`e settato a puntare ad uno stack temporaneo utilizzato dal gestore dei processi. Il 
salvataggio dei registri e il settaggio dello stack pointer devono essere svolti da una piccola routine in assembly, uguale per ogni interrupt. Quando la routine \`e finita chiama una
procedura in $C$ che svolge il resto del lavoro specifico al tipo di interrupt. Quando \`e finito viene chiamato lo scheduler per vedere quale processo deve essere eseguito. Dopo
quello il controllo \`e passato al codice in assembly per ricaricare in memoria i registri e le mappe per il proceso corrente e comincia ad essere eseguito. Dopo ogni interrupt il
processo ritorna precisamente nello stesso stato in cui era prima che accadesse l'interrupt. Riassumendo il processo di interrupt:
\begin{itemize}
	\item L'hardware salva lo stato del processo sullo stack.
	\item L'hardware carica il nuovo program counter dall'interrupt vector.
	\item Una procedura in assembly salva i registri.
	\item Una procedura crea un nuovo stack.
	\item Una servizio di interrupt in C viene eseguito (tipicamente legge e buffera l'input).
	\item Lo scheduler decide il prossimo processo da eseguire.
	\item Una procedura in C ritorna il codice in assmbly.
	\item Una procedura in assembly ricomincia il nuovo processo corrente.
\end{itemize}
\subsection{Modellare la multiprogrammazione}
L'utilizzo della multiprogrammazione permette il miglioramento dell'utilizzo della CPU: se il processo medio computa il $20\%$ del tempo che risiede in memoria, $5$ processi alla volta
dovrebbero rendere la CPU occupata tutti il tempo (irrealisticamente ottimistico). Un modello migliore consiste nel guardare l'utilizzo della CPU in modo probabilistico: suppondendo che
un processo utilizzi una frazione $p$ del suo tempo aspettando per il completamento dell'I/O. Con $n$ processi in memoria alla volta, la probabilit\`a che tutti gli $n$ processi stiano
aspettando per l'I/O \`e $p^n$, pertanto l'utilizzo della CPU \`e $1-p^n$. \`E facile notare come siano richiesti molti processi per rendere efficiente l'utilizzo della CPU. Si devono
naturalmente fare delle assunzioni: tutti i processi sono indipendenti mentre con una singola CPU non si possono avere processi che vengono svolti in contemporanea. Pur non essendo 
preciso, questo modello d\`a una buona approssimazione delle prestazioni della CPU. 
\section{Threads}
