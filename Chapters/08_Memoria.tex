\chapter{Gestione della memoria}
La condivisione della memoria da parte di pi\`u processi \`e essenziale per l'efficienza del sistema. Offre problematiche negli ambiti di allocazione della memoria ai singoli job, per
la protezione e condivisione dello spazio di indirizzamento e per la gestione della memoria virtuale (swap). Nei sistemi moderni la gestione della memoria \`e inseparabile dal concetto
di memoria virtuale. Ogni processo deve essere portato in memoria e trasformato in processo per essere eseguito. La CPU preleva le istruzioni da eseguire dalla memoria in base al
valore del program counter. L'istruzione viene codificata e pu\`o prevedere il prelievo di operandi dalla memoria. Al termine dell'istruzione il risultato pu\`o essere scritto in 
memoria. Quando il processo termina la sua memoria viene rilasciata. 
\section{Da programma a processo}
La trasformazione da programma a processo avviene attraverso varie fasi precedenti all'esecuzione: in ogni fase si ha una diversa semantica degli indirizzi (spazio logico e spazio
fisico). Gli indirizzi del programma sorgente sono simbolici e trasformati in indirizzi fisici attraverso il compilatore che associa agli indirizzi simbolici indirizzi rilocabili
e il linker o il loader associano agli indirizzi rilocabili indirizzi assoluti. Si noti come gli indirizzi hanno diverse rappresentazioni nelle varie fasi di costruzione di un programma.
Il collegamento tra indirizzi simbolici e fisici viene detto binding.
\subsection{Binding}
Il binding di dati e istruzioni a indirizzi di memoria pu\`o avvenire in tre momenti distinti:
\begin{enumerate}
	\item Tempo di compilazione: statico, se \`e noto a priori in quale parte della memoria risieder\`a il processo \`e possibile generare codice assoluto, ma se la locazione di 
		partenza cambia sar\`a necessario ricompilare.
	\item Tempo di caricamento: statico, si rende necessario generare codice rilocabile con indirizzi relativi all'inizio del programma. Se cambia indirizzo di riferimento si deve
		ricaricare. 
	\item Tempo di esecuzione: dinamico, il binding viene posticipato se il processo pu\`o essere spostato durante l'esecuzione in posizioni diverse della memoria. Viene richiesto 
		supporto hardware affinch\`e l'operazione possa essere fatta efficientemente. 
\end{enumerate}
\subsection{Collegamento}
Il collegamento pu\`o essere statico in cui tutti i riferimenti sono definiti prima dell'esecuzione e l'immagine del processo contiene una copia delle librerie usate o dinamico in cui il
link viene posticipato al tempo di esecuzione e il codice del programma non contiene il codice delle librerie ma solo un riferimento (stub) per poterle recuperare. 
\subsection{Caricamento}
Il caricamento pu\`o essere statico in cui tutto il codice viene caricato in memoria a tempo dell'esecuzione o dinamico in cui il caricamento dei moduli viene posticipato in 
corrispondenza del primo utilizzo (risulta utile per codice con molti casi speciali).
\subsection{Spazi di indirizzamento}
Lo spazio di indirizzamento logico \`e legato a uno spazio di indirizzamento fisico: l'indirizzo logico (o virtuale) \`e generato dalla CPU mentre quello fisico viene considerato dalla
memoria. Nel binding a compile o load-time l'indirizzo fisico e logico coincidono mentre in quello a run-time sono generalmente diversi. 
\subsubsection{Memory management unit (MMU)}
La memory management unit \`e un dispositivo hardware che mappa indirizzi virtuali in indirizzi fisici: il valore del registro di rilocazione \`e aggiunto ad ogni indirizzo generato
da un processo e inviato alla memoria. 
\subsection{Considerazioni}
In un sistema multiprogrammato non \`e possibile conoscere in anticipo dove un processo pu\`o essere posizionato in memoria e lo swap impedisce di poter utilizzare indirizzi rilocati
in modo statico. Pertanto non \`e possibile il binding a tempo di compilazione o di caricamento. Si deve far affidamento alla rilocazione dinamica usata per sistemi complessi come la
gestione della memoria nel sistema operativo o la rilocazione statica, possibile in sistemi progettati per applicazioni specifiche e con limitata gestione della memoria nel sistema
operativo. 
\section{Schemi di gestione della memoria}
I metodi principali di gestione della memoria sono:
\begin{itemize}
	\item Allocazione contigua.
	\item Paginazione.
	\item Segmentazione.
	\item Segmentazione doppia.
\end{itemize}
Si noti come tutti prevedono che il programma sia interamente caricato in memoria, mentre soluzioni realistiche usano memoria virtuale. 
\subsection{Allocazione contigua}
Nell'allocazione contigua i processi sono allocati in memoria in posizioni contigue all'interno di una partizione. Le partizioni possono essere fisse o variabili. La memoria \`e un 
insieme di partizioni di dimensioni predefinite e diverse. Nascono problematiche per l'assegnazione di memoria ai job e per il supporto della rilocazione dinamica. L'assegnazione della
memoria viene fatta dallo scheduling a lungo termine attraverso o una coda di partizione o una coda singola. 
\subsubsection{Assegnazione della memoria}
Se si trova una coda per partizione il processo viene assegnato alla partizione pi\`u piccola in grado di contenerlo. Si noti come \`e poco flessibile in quanto possono esserci 
partizioni vuote e job nelle altre code. Nel caso di una coda unica gestita con politica first come first served l'implementazione \`e facile ma vi \`e un basso utilizzo della memoria. 
La scansione della coda pu\`o avvenire attraverso best-fit-only in cui la scelta del job avviene scegliendo quello con le dimensioni pi\`u simili alla partizione o attraverso
first-available-fit in cui viene scelto il primo job che pu\`o stare nella partizione. 	
\subsubsection{Supporto per la rilocazione}
La MMU consiste di registri di rilocazione per proteggere lo spazio dei vari processi, attivamente e passivamente. Contiene il valore dell'indirizzo pi\`u basso (registro base o di 
rilocazione) e il limite superiore dello spazio logico (registro limite). Ogni indirizzo logico deve essere minore del limite.
\subsubsection{Considerazioni}
Si noti come questo \`e un approccio relativamente semplice ma il grado di multiprogrammazione \`e limitato dal numero di partizioni e nasce frammentazione che porta a spreco di memoria.
La frammentazione pu\`o essere interna: nella partizione se la dimensione della partizione \`e pi\`u grande della dimensione del job o esterna se vi sono partizioni non utilizzate
che non soddisfano le esigenze dei processi in attesa. 
\subsubsection{Tecnica delle partizioni variabili}
In questa tecnica lo spazio utente viene diviso in partizioni di dimensioni variabili identiche alla dimensione dei processi per eliminare la frammentazione interna. 
\paragraph{Assegnazione della memoria}
La memoria viene vista come un insieme di buche e il sistema operativo mantiene informazioni su partizioni allocate e buche. Quando arriva un processo gli viene allocata memoria usando
la buca che lo pu\`o contenere. Per soddisfare la richiesta di $n$ celle di memoria data una lista di buche libere si possono utilizzare le strategie:
\begin{itemize}
	\item First-fit: alloca la prima buca grande a sufficienza.
	\item Best-fit: alloca la pi\`u piccola buca grande a sufficienza. Richiede la scansione della lista e fornisce il minimo spreco.
	\item Worst-fit: alloca la buca pi\`u grande. Richiede la scansione della lista e lascia la buca di dimensioni pi\`u grandi.
\end{itemize}
Si noti come first fit \`e tipicamente la migliore. 	
\paragraph{Supporto per la rilocazione}
Come per le partizioni fisse i registri di rilocazione vengono usati per proteggere lo spazio dei vari processi attivamente e passivamente. 
\paragraph{Considerazioni}
Non fornisce frammentazione interna per costruzione ma la frammentazione esterna causa spreco di memoria in quanto nonostante esista lo spazio disponibile non \`e contiguo. Con first fit
dati $N$ blocchi allocati $0.5\cdot N$ blocchi vanno persi. La frammentazione si pu\`o ridurre attraverso compattazione: il contenuto della memoria viene spostato in modo da rendere
contigue tutte le partizioni. \`E possibile solo se la rilocazione \`e dinamica e la modifica del registro base. 
\subsubsection{Tecnica del buddy system}
In questa tecnica si deve fare il compromesso tra le partizioni fisse e variabili. La memoria viene vista come una serie di liste di blocchi di dimensione $2^k$ con $L < k < U$, dove
$2^L$ \`e il pi\`u piccolo blocco allocato e $2^U$ \`e il pi\`u grande blocco allocato. La memoria \`e disponibile sotto forma di blocchi di dimension $2^k$. All'inizio tutta la memoria
\`e disponibile: a lista di blocchi di dimensione $2^U$ contiene un solo blocco che rappresenta tutta la memoria mentre le altre sono vuote. Quando arriva una richiesta di dimensione $s$
si cerca un blocco libero con dimensione adatta purch\`e sia pari a una potenza del $2$. Se $2^{U-1} < s < 2^U$ l'intero blocco di dimensione $2^U$ viene allocato. Altrimenti il blocco 
$2^U$ viene diviso in due blocchi di dimensione $2^{U-1}$. Questa operazione viene ripetuta fino a che \`e viene allocato il processo o si arriva al blocco di dimensione $2^L$. Quando
un processo rilascia la memoria il suo blocco torna a far parte della lista dei blocchi di dimensione corrispondente. Se si formano $2$ blocchi adiacenti di dimensione $2^k$ \`e 
possibile compattarli ottenendo un unico blocco libero di dimensione $2^{k+1}$. Il vantaggio \`e che la compattazione richiede solo di scorrere la lista dei blocchi di dimensione $2^k$
ed \`e veloce, ma nasce la frammentazione interna dovuta solo ai blocchi di dimensione $2^L$. 
\section{Paginazione}
La paginazione nasce per eliminare la frammentazione esterna. Si permette che lo spazio di indirizzamento fisico di un processo sia non-contiguo. Si alloca memoria fisica dove essa \`e
disponibile. La memoria fisica viene divisa in blocchi di dimensione fissa detti frame e la memoria logica viene divisa in blocchi della stessa dimensione detti pagine. Per eseguire
un programma avente dimensione $n$ pagine bisogna trovare $n$ frame liberi. Si utilizza una tabella delle pagine (page table) per mantenere traccia di quale frame corrisponde a quale
pagina. Esiste una tabella delle pagine per ogni processo e viene usata per tradurre un indirizzo logico in un indirizzo fisico. La frammentazione interna nasce solo nell'ultima pagina. 
\subsection{Traduzione degli indirizzi}
L'indirizzo generato dalla CPU viene diviso in due parti: 
\begin{itemize}
	\item Numero di pagina (p): usato come indice nella tabella delle pagine che contiene l'indirizzo di base di ogni frame. 
	\item Offset (d): combinato con l'indirizzo base definisce l'indirizzo fisico che viene inviato alla memoria. Se la dimensione della memoria \`e $2^m$ e quella di una pagina \`e
		$2^n$ parole per byte i primi $m-n$ bit sono il numero della pagina e i successivi $n$ il numero di offset. 
\end{itemize}
\subsubsection{Implementazione della tabella delle pagine}
L'efficienza \`e fondamentale e pertanto la tabella si pu\`o implementare attraverso registro o implementazione in memoria (tabella multilivello o invertita).
\paragraph{Implementazione tramite registri}
Le entry (righe) della tabella delle pagine sono mantenute nei registri. La soluzione \`e efficiente ma fattibile solo se il numero di entry \`e limitato e allunga i tempi del context
switch in quanto richiede il salvataggio dei registri. 
\paragraph{Implementazione in memoria}
La tabella risiede in memoria e vengono utilizzati due registri: il page-table base register (PTBR) che punta alla tabella delle pagine e l'opzionale page-table length register (PTLR) 
che contiene la dimensione della tabella delle pagine. Il context switch \`e pi\`u breve in quanto richiede la modifica solo del PTBR (PTLR), ma ad ogni accesso a dati o istruzioni 
richiede due accessi in memoria per la tabella delle pagine e il risultante dato/istruzione. Il problema del doppio accesso pu\`o essere risolto tramite una cache molto veloce
detta translation look-aside buffers (TLB). Funzionano confrontando l'elemento fornico con il cambio chiave di tutte le entry contemporaneamente e nella tabella delle pagine la
chiave d\`a il numero di pagina e il valore il numero di frame. Essendo il TLB molto costoso viene memorizzato solo un piccolo sottoinsieme delle entry della tabella delle pagine. Ad
ogni context switch il TLB viene ripulito per evitare il mapping di indirizzi errati. Durante un accesso alla memoria se la pagina cercata \`e nel TLB questo restituisce il numero di 
frame con un singolo accesso (minore del $10\%$ del tempo richiesto in assenza di TLB), altrimenti \`e necessario accedere alla tabella delle pagine in memoria. L'hit ratio $\alpha$ \`e
la percentuale delle volte in cui una pagina si trova nel TLB. Si rende necessario definire il concetto di tempo di accesso effettivo: 
$$EAT = (T_{MEM} + T_{TLB})\cdot\alpha + (2\cdot T_{MEM} + T_{TLB})\cdot (1-\alpha)$$
Dove $T_{TLB}$ \`e il tempo di accesso a TLB e $T_{MEM}$ \`e il tempo di accesso a memoria.
\subsubsection{Protezione}
La protezione viene associata associando bit di protezione ad ogni livello come il bit di validit\`a (valid-invalid bit) per ogni entry della tabella delle pagine che risulta utile per
la memoria virtuale:
\begin{itemize}
	\item Valid: la pagina associata \`e nello spazio di indirizzamento logico del processo.
	\item Invalid: la pagina associata non \`e nello spazio di indirizzamento logico del processo.
\end{itemize}
Bit di accesso:
\begin{itemize}
	\item Per marcare una pagina modificabile o meno (read-only).
	\item Per marcare una pagina eseguibile o meno. 
\end{itemize}
\subsubsection{Pagine condivise}
Per il codice condiviso si trova un'unica copia fisica ma pi\`u copie logiche (una per processo), il codice read-only rientrante che non cambia mai durante l'esecuzione pu\`o essere
condiviso tra processo. I dati in generale saranno diversi da processo a processo con pi\`u copie fisiche e logiche. 
\subsubsection{Spazio di indirizzamento}
Nelle architetture moderne lo spazio di indirizzamento virtuale \`e molto maggiore dello spazio fisico. Sono necessari meccanismi per gestire il problema della dimensione della
tabella delle pagine attraverso la paginazione della tabella delle pagine (tabella multilivello) o la tabella delle pagine invertita. 
\paragraph{Pagine multilivello}
Questo metodo \`e equivalente a paginare la tabella delle pagine: solo alcune parti della tabella sono memorizzate in memoria, altre si trovano sul disco. Sono possibili versioni da
$2$ a $4$ livelli. Ogni livello \`e memorizzato come una tabella separata in memoria, la conversione dell'indirizzo logico in quello fisico pu\`o richiedere anche $4$ accessi a memoria
e il TLB mantiene le prestazioni a livelli ragionevoli. 
\paragraph{Tabella delle pagine invertita}
Si trova una tabella unica nel sistema con un'entry per ogni frame contenente la coppia $<$process-id, page-number$>$ dove il primo \`e l'identificativo del processo che possiede la
pagina e il secondo \`e l'indirizzo logico della pagina contenuta nel frame corrispondente a quella entry. Ogni indirizzo logico generato dalla CPU \`e una tripla: $<$process-id, 
page-number, offset$>$. \`E necessario cercare il valore desiderato e pertanto un aumento del tempo necessario per cercare un riferimento ad una pagina. Per questioni di ricerca si
deve ridurre il tempo di ricerca a $O(1)$ attraverso una tabella hash e si crea la necessit\`a di un meccanismo per gestire le collisioni quando diversi indirizzi virtuali corrispondono
allo stesso frame. 
\section{Segmentazione}
