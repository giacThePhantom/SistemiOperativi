\chapter{File System}
Il file system fornisce il meccanismo per la memorizzazione e l'accesso di dati e programmi. Consiste di una collezione di file e della struttura di cartelle (directory)
\section{Interfaccia del file system}
\subsection{Concetto di file}
Il sistema operativo astrae dalle caratteristiche fisiche dei supporti di memorizzazione fornendo una loro visione logica. Si considera un file come uno spazio di indirizzamento logico 
e contiguo rappresentante un insieme di informazioni correlate identificate da un nome. I file possono essere dati di qualsiasi tipo e programmi. 
\subsection{Attributi di un file}
Su disco nella struttura della directory sono salvati per ogni file il nome (unica informazione in formato leggibile), tipo, posizione (puntatore allo spazio fisico sul dispositivo), 
dimensione, protezione (controllo sui permessi), tempo, data e identificazione dell'utente. 
\subsection{Operazioni sui file}
\begin{itemize}
	\item Creazione: si cerca lo spazio necessario su disco e si crea un nuovo elemento sulla directory per gli attributi.
	\item Scrittura: una system call che specifica il nome del file e i dati da scrivere, \`e necessario che sia noto il puntatore alla locazione della prossima scrittura.
	\item Lettura: una system call che specifica il nome del file e dove mettere i dati letti in memoria, \`e necessario conoscere il puntatore alla locazione della prossima
		lettura, lo stesso della scrittura. 
	\item Riposizionamento all'interno di file: aggiornamento del puntatore alla posizione corrente.
	\item Cancellazione: libera lo spazio associato al file e l'elemento corrispondente nella directory. 
	\item Troncamento: mantiene inalterati gli attributi ma cancella il contenuto del file. 
	\item Apertura: ricerca il file nella struttura della directory su disco, copia il file in memoria e inserisce un riferimento nella tabella dei file aperti. In sistemi 
		multi utente si trovano $2$ tabelle: una per ogni processo che contiene i riferimenti per i file aperti relativi a esso e una per tutti i file aperti da tutti i 
		processi che contiene i dati indipendenti dal processo. 
	\item Chiusura: copia del file in memoria su disco. 
\end{itemize}
\subsection{Struttura dei file}
I tipi di file possono essere usati per indicare la struttura interna del file. In Unix non esiste, si considera un file come una sequenza di parole o bytes. Pu\`o essere una struttura a 
record semplice dove un record \`e una riga di lunghezza fissa o variabile. Pu\`o essere una struttura complessa come accade per i documenti formattati e i formati ricaricabili (load
module). Le ultime due si possono emulare con la prima usando caratteri di controllo. 
\subsection{Metodi di accesso}
\subsubsection{Sequenziale}
Questo metodo viene usato da editor e compilatori, permettono le operazioni \emph{read next}, \emph{write next}, \emph{reset} (rewind). Non \`e permesso il rewrite in quanto si rischia
inconsistenza se si scrive qualcosa a met\`a del file in quanto si potrebbe eliminare ci\`o che sta dopo. 
\subsubsection{Diretto}
Come accade nel database in cui un file \`e una sequenza numerata di blocchi detti record. Le operazioni permesse sono \emph{read n}, \emph{write n}, \emph{position to n}, \emph{read
next}, \emph{write next}, \emph{rewrite n}. 
\section{Struttura delle directory}
Le directory sono una collezione di nodi contenente informazioni sui file. Si trovano entrambi sul disco e determinano la struttura del file system. Per ogni file contengono il 
nome, tipo, indirizzo, lunghezza attuale, massima lunghezza, data di ultimo accesso, data di ultima modifica, proprietario e informazioni di protezione. 
\subsection{Operazioni sulla directory}
\begin{itemize}
	\item Aggiungere un file.
	\item Cancellare un file.
	\item Visualizzare il contenuto della directory.
	\item Rinominare un file.
	\item Cercare un file.
	\item Attraversare il file system. 
\end{itemize}
\subsection{Organizzazione logica}
Gli obiettivi dell'organizzazione logica delle directory sono l'efficienza (rapido accesso ad un file), nomenclatura conveniente agli utenti in quanto permette stessi nomi per file
e utenti diversi e nomi diversi per lo stesso file e raggruppamento: i file vengono classificati logicamente per un criterio come il tipo o la protezione. 
\subsubsection{Directory a un livello}
In questo metodo si trova una singola directory per tutti gli utenti. Nascono problemi di nomenclatura in quanto \`e difficile ricordare se un nome esiste gi\`a e inventarne sempre di
nuovi. Nascono inoltre problemi di raggruppamento. 
\subsubsection{Directory a due livelli}
In questo metodo si trova una directory separata per ogni utente. Nasce il concetto di percorso (path), \`e possibile usare lo stesso nome di file per utenti diversi e la ricerca \`e
efficiente. Non esiste comunque raggruppamento e si pone il problema di dove mettere i programmi di sistema condivisi dagli utenti (in Unix si usa \emph{PATH}).
\subsubsection{Directory ad albero}
In questo metodo la directory viene implementata come un albero permettendo una ricerca efficiente e la possibilit\`a di raggruppamento. Nasce il concetto di working directory e i
nomi di percorso possono essere assoluti o relativi. 
\subsubsection{Directory a grafo aciclico}
In questo metodo si espande la versione ad albero a grafo aciclico per permettere la condivisione di file. La condivisione avviene in Windows attraverso i collegamenti e in Unix
attraverso i link. I link possono essere simbolici: contengono il pathname del file reale e se si cancella il file il link rimane pendente o hard: si trova un contatore che mantiene
il numero di riferimenti decrementato per ogni cancellazione di un riferimento. Il file pu\`o essere cancellato solo se il contatore vale $0$. 
\subsubsection{Directory a grafo generico}
Questo metodo nasce come estensione della directory a grafo aciclico e si deve garantire che le ricerche terminino per evitare loop infiniti nell'attraversamento del grafo. Per risolvere
questo problema si pu\`o permettere di collegare solo file non directory o usare un algoritmo di controllo di esistenza di un ciclo ogni volta che si crea un nuovo collegamento, che 
pu\`o essere costoso. 
\subsection{Mount di file system}
Si possono creare dei file system modulari con la possibilit\`a di attaccare (mount) e staccare (unmount) interi file system a file system precedenti. In generale un file system deve
essere montato prima di potervi accedere. Ci si riferisce al punto in cui viene montato come al mount point. 
\subsection{Condivisione di file}
La condivisione di file \`e importante in sistemi multiutente ed \`e realizzabile tramite uno schema di protezione. Nei sistemi distribuiti i file possono essere condivisi attraverso
una rete. Il Network File System (NFS) \`e un tipico schema di condivisione di file via rete. 
\subsection{Protezione}
Il possessore di un file deve poter controllare cosa \`e possibile fare su un file e da parte di chi. Per farlo si pu\`o implementare una lista di accesso per ogni file o directory
che lista chi pu\`o fare cosa. Pu\`o essere lunga. Gli utenti si possono raggruppare in tre classi con una perdita di generalit\`a: il proprietario, il gruppo (utenti appartenenti allo
stesso gruppo del proprietario) e gli altri. Per ogni classe i permessi possono essere di lettura, scrittura o esecuzione. 
\section{Implementazione del file system}
Per gestire un file system si usano diverse strutture dati in parte su disco e in parte in memoria. Le caratteristiche, pur avendo una base comune sono fortemente dipendenti dal 
sistema operativo e dal tipo di file system. 
\subsection{Strutture su disco}
\begin{itemize}
	\item Blocco di boot: contiene le informazioni necessarie per l'avvio del sistema operativo.
	\item Blocco di controllo delle partizioni: contiene i dettagli riguardanti la partizione come numero e dimensione dei blocchi, la lista dei blocchi liberi, dei descrittori 
		liberi.
	\item Strutture di directory che descrivono l'organizzazione dei file.
	\item Descrittori dei file che contengono vari dettagli sui file e i puntatori ai blocchi dati. 
\end{itemize}
\subsection{Strutture in memoria}
\begin{itemize}
	\item Tabella delle partizioni: informazioni sulle partizioni montate.
	\item Strutture di directory: copia in memoria delle directory a cui si \`e fatto accesso di recente.
	\item Tabella globale dei file aperti con le copie dei descrittori dei file.
	\item Tabella dei file aperti per ogni processo contenente il puntatore alla tabella globale e le informazioni di accesso. 
\end{itemize}
\subsection{Allocazione dello spazio su disco}
I blocchi su disco possono essere allocati ai file o alle directory in modi diversi con l'obiettivo di minimizzare i tempi di accesso e massimizzare l'utilizzo dello spazio. 
\subsubsection{Allocazione contigua}
In questo metodo ogni file occupa un insieme di blocchi contigui su disco. L'entry della directory \`e semplice in quanto contiene l'indirizzo del blocco di partenza e la lunghezza del
file. Permette un accesso semplice in quanto l'accesso al blocco $b+1$ non richiede lo spostamento della testina rispetto al blocco $b$ a meno che $b$ non sia l'ultimo blocco di un 
cilindro. Supporta sia l'accesso sequenziale che casuale in quanto per leggere il blocco $i$ di un file che inizia al blocco $b$ basta leggere $b+i$. Presenta per\`o problemi 
simili a quelli dell'allocazione dinamica: si deve scegliere tra algoritmi best-fit, first-fit o worst-fit e porta a uno spreco di spazio dovuto a frammentazione esterna e richiede
una compattazione periodica dello spazio. La decisione della dimensione dello spazio da allocare \`e un problema in quanto i file potrebbero crescere dinamicamente. Si pu\`o fare in
modo che se il file deve crescere e non c'\`e spazio nasce un'errore e una terminazione del programma e si riesegue. In questo modo si tende a sovrastimare lo spazio necessario che
porta a uno spreco. Un'altra soluzione consiste di trovare un buco pi\`u grande e ricopiare tutto il file in questo, operazione trasparente per l'utente ma che rallenta il sistema. 
\paragraph{Variante}
Alcuni file system moderni usano uno schema modificato di allocazione contigua come Linux ext2fs basato sull'extent, una serie di blocchi contigui su disco. Il file system alloca
extent invece di blocchi singoli che in generale non sono contigui e si prestano pertanto per l'allocazione a lista.
\subsubsection{Allocazione a lista}
In questo metodo ogni file si trova su una lista di blocchi che possono essere sparsi ovunque nel disco. La directory contiene puntatori al primo e all'ultimo blocco e ogni blocco
contiene un puntatore al blocco successivo. Considerando $X$ l'indirizzo logico e $N$ la dimensione del blocco $\frac{X}{N-1}$ \`e il numero del blocco nella lista, mentre
$X\mod (n - 1)$ \`e l'offset all'interno del blocco. L'allocazione a lista permette una semplice creazione di un nuovo file in quanto basta cercare un blocco libero e creare una
nuova entry nella directory che punta ad esso. Anche l'estensione del file \`e semplice: si cerca un blocco libero e lo si concatena alla fine del file. Non c'\`e  spreco se non
si considera lo spazio per il puntatore in quanto si pu\`o usare qualunque blocco e non c'\`e frammentazione esterna. Non permette per\`o accesso casuale in quanto bisogna scorrere 
tutti i blocchi a partire dal primo e richiede tanti riposizionamenti sparsi che portano a scarsa efficienza. \`E inoltre scarsamente affidabile in quanto la perdita di un puntatore o 
un errore causa il prelevamento del puntatore sbagliato si devono introdurre metodi di recupero con overhead come le liste doppiamente concatenate e memorizzare il nome del file e il 
numero di blocco in ogni blocco del file. 
\paragraph{Esempio - FAT (file allocation table)} Si trova una FAT per ogni partizione che contiene un elemento per ogni blocco del disco usata come lista concatenata che migliora 
l'accesso causale, che rimane comunque con bassa efficienza. 
\subsubsection{Allocazione indicizzata}
In questo metodo ogni file ha un blocco indice (index block) che contiene la tabella degli indirizzi dei blocchi fisici. La directory contiene l'indirizzo del blocco indice. L'accesso
casuale \`e efficiente e quello dinamico \`e senza frammentazione esterna con overhead del blocco indice per la index table, maggiore di quello richiesto per l'allocazione concatenata. 
Se $X$ \`e l'indirizzo logico e $N$ la dimensione del blocco $\frac{X}{N}$ \`e l'offset nella index table e $X\mod N$ \`e l'offset all'interno del blocco dati. Si noti come la dimensione
del blocco limita la dimensione del file, pertanto per i file di grandi dimensioni si usa uno schema a pi\`u livelli.
\paragraph{Indici multilivello} Una tabella pi\`u esterna contiene puntatori ad ulteriori index table. Sia $X$ l'indirizzo logico e $N$ la dimensione del blocco in parole 
$\frac{X}{n\cdot N}$ \`e il blocco della index table di primo livello e $X\mod (N\cdot N) = R$, dove $\frac{R}{N}$ \`e l'offset nel blocco della index table di secondo livello e 
$R\mod N$ \`e l'offset nel blocco dati.
\paragraph{Schema concatenato} Si trova una lista concatenata di blocchi indice: l'ultimo degli indici di un blocco indice punta a un altro blocco indice e $\frac{X}{N(N-1)}$ \`e il 
numero del blocco indice all'interno della lista dei blocchi indice e $X\mod(N(N-1)) = R$, dove $\frac{R}{N}$ \`e l'offset nel blocco indice e $R\mod N$ \`e l'offset nel blocco dati. 
\subsubsection{Unix e i-node}
Unix usa lo schema combinato in cui ad ogni file \`e associato un i-node (index-node) che contiene gli attributi del file e fa da blocco indice. Gli i-node sono gestiti dal sistema 
operativo e memorizzati in modo permanente in una porzione riservata al sistema operativo di solito all'inizio del disco. Per memorizzare i blocchi di dati del file ogni i-node contiene:
\begin{itemize}
	\item $10$ puntatori diretti ai blocchi di dati di file.
	\item Un puntatore single indirect che punta ad un blocco indice che contiene puntatori a blocchi di dati di file.
	\item Un puntatore double indirect che punta ad un blocco indice che contiene puntatori a blocchi indice contenenti puntatori a blocchi di dati di file.
	\item Un puntatore triple indirect che punta ad un blocco indice che contiene puntatori a blocchi indice che contengono puntatori a blocchi indice che contengono puntatori a
		blocchi di dati di file. 
\end{itemize}
\subsection{Implementazione delle directory}
Le directory vengono implementate con lo stesso meccanismo usato per memorizzare i file. Non contengono file ma la lista di file e directory che contengono. Questo spazio pu\`o essere
memorizzato come lista lineare di nomi di file con puntatori ai blocchi dati di facile implementazione ma poco efficiente in quanto lettura, scrittura e rimozione di file richiedono 
ricerca per trovare il file. Pu\`o essere anche implementato come tabella hash con tempo di ricerca migliore ma che richiede la gestione delle collisioni. 
\section{Gestione dello spazio libero}
Per tenere traccia dello spazio libero su disco si mantiene una lista dei blocchi liberi. Per creare un file si cercano blocchi liberi nella lista e per rimuovere un file si aggiungono
i suoi blocchi a tale lista. 
\subsection{Alternative}
\begin{itemize}
	\item Vettore di bit: si trova un elemento per ogni blocco tale che $Bit[i] = 0$ implica che il blocco $i$ \`e libero e $Bit[i] = 1$ implica che \`e occupato. La mappa di bit
		richiede spazio extra ed \`e efficiente solo se il vettore \`e mantenibile tutto in memoria. Risulta comunque facile ottenere blocchi contigui. 
	\item Lista concatenata di blocchi liberi (free list): permette spreco minimo in testa alla lista e lo spazio contiguo non \`e ottenibile.
	\item Raggruppamento: modifica la lista linkata in modo che il primo blocco libero contiene gli indirizzi di $n-1$ blocchi liberi e l'ultima entry del blocco l'indirizzo del
		primo blocco del gruppo successivo di blocchi liberi. Fornisce rapidamente un gran numero di blocchi liberi. 
	\item Conteggio: mantiene il conteggio di quanti blocchi liberi seguono il primo in una zona di blocchi liberi contigui. La lista risulta pi\`u corta se il contatore \`e maggiore
		di $1$ per ogni gruppo di blocchi liberi. 
\end{itemize}
\section{Efficienza e prestazioni}
\subsection{Efficienza}
Il disco \`e un collo di bottiglia e l'efficienza dipende da: 1) un algoritmo di allocazione dello spazio su disco: in Unix si cerca di tenere i blocchi di un file vicini al suo i-node, 
viene richiesta la preallocazione degli i-node distribuiti sulla partizione; 2) dal tipo di dati contenuti nella directory: ad esempio la data di ultimo accesso di un file richiede che la lettura di un
file richieda la lettura e scrittura del blocco della directory. 
\subsection{Prestazioni}
Il controller del disco possiede una piccola cache che \`e in grado di contenere un'intera traccia ma non basta per garantire prestazioni elevate pertanto si usano dischi virtuali (RAM
disk) e cache del disco o buffer cache
\subsubsection{Dischi virtuali}
Parte della memoria viene gestita come se fosse un disco: il driver di un RAM disk accetta tutte le operazioni standard dei dischi eseguendole in memoria. \`E veloce ma supporta solo
file temporanei. Viene gestito dall'utente che scrive sul RAM disk invece che sul disco. 
\subsubsection{Cache del disco}
\`E una porzione di memoria che memorizza blocchi usati di frequente simile alla cache tra memoria e CPU. Viene gestita dal sistema operativo e sfrutta il principio di localit\`a
spaziale e temporale. Il trasferimento di dati nella memoria del processo utente non richiede spostamento di byte. Nascono problematiche riguardanti la politica di rimpiazzamento
attraverso LRU (poco efficiente, meglio rilascio indietro e lettura anticipata), LFU, RANDOM. Se l'operazione \`e una scrittura si pu\`o fare write-back: si scrive solo quando si 
deve rimuovere il blocco dalla cache (pu\`o generare problemi di affidabilit\`a in caso di crash) o write-through: si scrive sempre, meno efficiente con la cache in sola lettura. 
\subsubsection{Recupero}
I problemi di consistenza tra disco e cache possono essere controllati attraverso un controllo di consistenza che confronta i dati nella directory con i dati su disco e sistema le
inconsistenze o attraverso l'uso di programmi di sistema per fare il backup del disco su memoria di massa come i nastri che permettono il recupero di file persi tramite il restore
dei dati di backup. 
\subsubsection{File system log structured}
Si registra ogni cambiamento del file system come una transazione. Tutte le transazioni sono scritte su log e considerate completate quando viene scritta su esso (anche se il file
system pu\`o non essere aggiornato). Le transazioni sono scritte in modo asincrono nel file system e quando il file system \`e modificato la transazione viene cancellata dal log. Se
il sistema va in crash le transazioni non avvenute sono quelle presenti sul log. In questo modo si ottimizza il numero di seek. 
